\documentclass[12pt]{article}
\usepackage{amsmath,amssymb,amsthm,geometry}
\geometry{margin=1in}

\title{Proof of Convergence for the Greedy Voting Mechanism}
\author{}
\date{}

\begin{document}

\maketitle

\section*{Greedy Proof}

\subsection*{Definitions}

Let 
\[
V_k := \text{set of agents left at step } k,
\]
with \( V_1 = V \). Define the update rules
\[
V_{i+1} = V_i \cap C_i^c, 
\qquad 
S_{i+1} = S_i \cap \text{(expired choices)}^c,
\]
where \( C_i \) is the set of agents whose preferences were approved on step \( i \).

We aim to prove
\[
\lim_{n \to \infty} \Pr\big(\text{Greedy}(V) \in \operatorname{argmax} Q(S)\big) = 1.
\]

\subsection*{Base Case: \( k = 1 \)}

Since \( V_1 = V \), it follows that
\[
\text{Greedy}(V) = \text{AV}(V),
\]
by identical process at the first step. Hence, the proposition holds for \( k = 1 \).

\subsection*{Inductive Step}

Assume the proposition holds for some step \( k \). We consider two cases:

\paragraph{Case 1: \( V_{k+1} \neq \varnothing \).}  
Then
\[
\text{Greedy}(V_{k+1}) = \text{AV}(V_{k+1}, S_{k+1}),
\]
which, by the inductive hypothesis, satisfies the proposition.

\paragraph{Case 2: \( V_{k+1} = \varnothing \).}  
Then
\[
\text{Greedy}(S_{k+1}) = \text{Uniform}(S_{k+1}),
\]
which does not satisfy the property above. We will show that the probability of this case tends to zero as \( n \to \infty \).

\subsection*{Lemma 1}

\begin{lemma}
\[
\Pr(\text{Case 2 happens}) \to 0 \quad \text{as } n \to \infty.
\]
\end{lemma}

\begin{proof}
Let \( p_i \) denote the probability that agent \( i \)’s preference is approved at some step, with \( p_i \in (0,1) \).
Define
\[
p^* := \max_{i \in \{1, \ldots, |S|\}} p_i.
\]

The probability that all agents’ approvals vanish (i.e. \( V_m = \varnothing \)) is given by the union of the disjoint events that each agent fails to approve at all steps:
\[
\Pr(V_m = \varnothing) = \Pr\left(\bigcup_{i=1}^{|S|} \text{agent } i \text{ unapproved}\right)
\leq \sum_{i=1}^{|S|} (1 - p_i)^n.
\]

Since \( (1 - p_i)^n \leq (1 - p^*)^n \) for all \( i \),
\[
\Pr(V_m = \varnothing)
\leq |S|(1 - p^*)^n.
\]

Taking the limit,
\[
\lim_{n \to \infty} \Pr(V_m = \varnothing)
\leq \lim_{n \to \infty} |S|(1 - p^*)^n = 0.
\]
Thus, as \( n \to \infty \), Case 2 occurs with probability \( 0 \).
\end{proof}

\subsection*{Lemma 2}

\begin{lemma}
For \( k = 1 \),
\[
\text{GreedyHelper}(S, V, k=1) = \text{ApprovalVoting}(S, V, k=1).
\]
\end{lemma}

\begin{proof}
By definition, at \( k = 1 \), the \texttt{GreedyHelper} procedure selects the element 
\[
s^* = \arg\max_{s \in S} \sum_{v \in V} \mathbf{1}\{s \in \text{pref}(v)\},
\]
that is, the candidate \( s \in S \) receiving the largest number of approvals from agents in \( V \).

The Approval Voting mechanism likewise chooses
\[
s_{\text{AV}} = \arg\max_{s \in S} \sum_{v \in V} \mathbf{1}\{s \in \text{pref}(v)\}.
\]

Therefore, for \( k = 1 \), the two mechanisms are equivalent:
\[
\text{GreedyHelper}(S, V, 1) = \text{ApprovalVoting}(S, V, 1).
\]
\end{proof}

\subsection*{Conclusion}

By Lemma 1, as \( n \to \infty \), the inductive step proceeds via Case 1 with probability \( 1 \).  
By Lemma 2 and the base case, the Greedy mechanism coincides with Approval Voting at the first step.

Therefore, by the \textit{Principle of Mathematical Induction},
\[
\lim_{n \to \infty} \Pr\big(\text{Greedy}(V) \in \operatorname{argmax} Q(S)\big) = 1.
\]
\hfill\(\Box\)

\end{document}
